\documentclass[10pt,twocolumn,letterpaper]{article}

\usepackage{cvpr}
\usepackage{times}
\usepackage{epsfig}
\usepackage{graphicx}
\usepackage{amsmath}
\usepackage{amssymb}

% Include other packages here, before hyperref.

% If you comment hyperref and then uncomment it, you should delete
% egpaper.aux before re-running latex.  (Or just hit 'q' on the first latex
% run, let it finish, and you should be clear).
\usepackage[breaklinks=true,bookmarks=false]{hyperref}

\cvprfinalcopy % *** Uncomment this line for the final submission

\def\cvprPaperID{****} % *** Enter the CVPR Paper ID here
\def\httilde{\mbox{\tt\raisebox{-.5ex}{\symbol{126}}}}

% Pages are numbered in submission mode, and unnumbered in camera-ready
%\ifcvprfinal\pagestyle{empty}\fi
\setcounter{page}{1}
\begin{document}

%%%%%%%%% TITLE
\title{VCS 2020: Locate and Recognize Paintings and People}

\author{Roberto Amoroso\\
University of Modena and Reggio Emilia\\
{\tt\small 219620@studenti.unimore.it}
}

\maketitle
%\thispagestyle{empty}

%%%%%%%%% ABSTRACT
\begin{abstract}
   We propose a method to locate and recognize paintings and people in a museum or art gallery. For this purpose, we created a Python program that is able to locate and recognize paintings and people present in a video or single image. For the part relating to the paintings, we used the OpenCV library, while to carry out the people detection operation we used YOLO, a real-time object detection system.
\end{abstract}

%%%%%%%%% BODY TEXT
\section{Introduction}
Localizzare un dipinto, calcolare la traformazione che consente di rettificare l'immagine e poi confrontare l'immagine ottenuta con quelle memorizza in un datase, sono tutte operaioni nontrivial. 

Le tecniche comuni di image processing soffrono di alcuni problemi (illuminazione, scale changes, distorsione, etc.) che diventano ancora più rilevanti nel caso di moving cameras, le quali a loro volta introducono ulteriori effetti indesiderati, e.g. blur, noise, motion. 
In aggiunta a ciò, l'esposione dei dipinti in esibizioni e mostre aggiunge problemi di riflessione delle luci, image exposure and image saturation.

Negli ultimi anni, sono stati sviluppati molti lavori che investigano i problemi di image detection \cite{fathy1995image,hambly2001supercosmos}, recognition \cite{martinel2013robust} and retieval \cite{rui1999image}. Molti di questi lavori, match le appearance features di un dipinto against a large database of location-tagged images.


Il problema consiste nel localizzare e riconoscere dipinti e persone in un museo o galleria d'arte.
Ogni dipinto localizzato viene confrontato con un database di immagini di dipinti a nostra disposizione, così da ricavare tutte le informazioni ad esso correlate.

%------------------------------------------------------------------------
\section{Related Work}
-- TODO --
Inserisci informazioni riguardo a ciascuno dei punti obbligatori che hai sviluppato

%------------------------------------------------------------------------
\section{Methods}
-- TODO --
%------------------------------------------------------------------------
\subsection{Data Collection and Preparation}
-- TODO --
%------------------------------------------------------------------------
\subsection{Painting Detection}
-- TODO --
%------------------------------------------------------------------------
\subsection{Painting Rectification}
-- TODO --
%------------------------------------------------------------------------
\subsection{Painting Retieval}
-- TODO --
%------------------------------------------------------------------------
\subsection{People Detection}
-- TODO --
%------------------------------------------------------------------------
\subsection{People and Painting Localization}
-- TODO --

%------------------------------------------------------------------------
\section{Results}
-- TODO --
Mostra i risultati complessivi e dove l'approccio fallisce quando non vengono rispettate le premesse e i presupposti fatti precedentemente (vedi ad esempio il flooding della parete).

%------------------------------------------------------------------------
\section{Conclusions}
-- TODO --

{\small
\bibliographystyle{ieee_fullname}
\bibliography{egbib}
}

\end{document}
